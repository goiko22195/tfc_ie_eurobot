%%%%%%%%%%%%%%%%%%%%%%%%%%%%%%%%%%%%%%%%%%%%%%%%%%%%%%%%%%%%%%%%%%%%%%%%%%%
%
% Generic template for TFC/TFM/TFG/Tesis
%
% $Id$
%
% By:
%  + Javier Mac�as-Guarasa.
%    Departamento de Electr�nica
%    Universidad de Alcal�
%  + Roberto Barra-Chicote.
%    Departamento de Ingenier�a Electr�nica
%    Universidad Polit�cnica de Madrid
%
% Based on original sources by Roberto Barra, Manuel Oca�a, Jes�s Nuevo,
% Pedro Revenga, Fernando Herr�nz and Noelia Hern�ndez. Thanks a lot to
% all of them, and to the many anonymous contributors found (thanks to
% google) that provided help in setting all this up.
%
% See also the additionalContributors.txt file to check the name of
% additional contributors to this work.
%
% If you think you can add pieces of relevant/useful examples,
% improvements, please contact us at (macias@depeca.uah.es)
%
% Copyleft 2013
%
%%%%%%%%%%%%%%%%%%%%%%%%%%%%%%%%%%%%%%%%%%%%%%%%%%%%%%%%%%%%%%%%%%%%%%%%%%%

\chapter{Sistema de medida de la posici�n del oponente}
\label{cha_sistema_baliza_faro}

\begin{FraseCelebre}
  \begin{Frase}
TODO
La imaginaci�n es m�s importante que el conocimiento. El conocimiento es limitado y la imaginaci�n circunda el mundo.    
  \end{Frase}
  \begin{Fuente}
Albert Einstein, en \emph{The Saturday Evening Post}
  \end{Fuente}
\end{FraseCelebre}

Respecto a la evitaci�n de obst�culos, esta se realiza a partir de sensores de distancia situados
en las caras del robot (ver figura 3.2) y a partir un sistema de balizas espec�ficamente desarrollado para
la medida de la posici�n del oponente. Este sistema hace uso de los soportes y espacios destinados a
sistemas de balizas. Concretamente, el sistema desarrollado est� basado en un sensor tipo faro situado en
el robot y balizas reflectantes situadas en el oponente (ver figura 3.2). El sistema de balizas es totalmente
aut�nomo y se comunica mediante Bluetooth con la tarjeta principal la plataforma rob�tica.

\section{Principio de funcionamiento}
\section{Desarrollo hardware y software}
\section{Caracterizaci�n y resultados}



%%% Local Variables:
%%% TeX-master: "../book"
%%% End:
