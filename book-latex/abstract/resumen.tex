%%%%%%%%%%%%%%%%%%%%%%%%%%%%%%%%%%%%%%%%%%%%%%%%%%%%%%%%%%%%%%%%%%%%%%%%%%%
%
% Generic template for TFC/TFM/TFG/Tesis
%
% $Id: resumen.tex,v 1.7 2014/01/08 22:56:02 macias Exp $
%
% By:
%  + Javier Mac�as-Guarasa. 
%    Departamento de Electr�nica
%    Universidad de Alcal�
%  + Roberto Barra-Chicote. 
%    Departamento de Ingenier�a Electr�nica
%    Universidad Polit�cnica de Madrid   
% 
% Based on original sources by Roberto Barra, Manuel Oca�a, Jes�s Nuevo,
% Pedro Revenga, Fernando Herr�nz and Noelia Hern�ndez. Thanks a lot to
% all of them, and to the many anonymous contributors found (thanks to
% google) that provided help in setting all this up.
%
% See also the additionalContributors.txt file to check the name of
% additional contributors to this work.
%
% If you think you can add pieces of relevant/useful examples,
% improvements, please contact us at (macias@depeca.uah.es)
%
% Copyleft 2013
%
%%%%%%%%%%%%%%%%%%%%%%%%%%%%%%%%%%%%%%%%%%%%%%%%%%%%%%%%%%%%%%%%%%%%%%%%%%%

\chapter*{Resumen}
\label{cha:resumen}
\markboth{Resumen}{Resumen}

\addcontentsline{toc}{chapter}{Resumen}

Este trabajo de fin de carrera trata sobre el desarrollo de robots para la competici�n de robots Eurobot \cite{eurobot}. Pretende ser un manual de referencia para todo aquel que quiera construir un robot para participar en Eurobot. El trabajo est� basado en la experiencia adquirida en el desarrollo de este tipo de robots entre entre los a�os 2003 y 2015. Especialmente en el periodo entre 2010 y 2015.

Eurobot es una competici�n internacional cuyas reglas y tem�tica el juego cambian cada a�o. El libro describe la mec�nica, hardware y software relativos al desarrollo e implementaci�n de las diferentes funcionalidades de un robot de Eurobot, las cuales se han dividido en dos partes principales: la \emph{plataforma rob�tica base} y los \emph{sistemas mec�nicos} de manipulaci�n de elementos del juego.

La plataforma rob�tica se basa en un \emph{sistema de tracci�n diferencial} que implementa un \emph{posicionamiento por odometr�a} y un \emph{control de posici�n polar}. La plataforma, adem�s, utiliza sensores para detectar obst�culos y un \emph{sistema de balizas} para la medir la posici�n de los robots oponentes, a partir del cual se implementa la \emph{evitaci�n de obst�culos}.

Mediante el estudio del modelo din�mico de la plataforma rob�tica se determinan los l�mites de aceleraci�n y velocidad, a partir de los cuales, dimensionar y especificar los elementos mec�nicos y estructura de la misma.

El desarrollo \emph{hardware} implementa una \emph{arquitectura} que permite ser reutilizada en el desarrollo de nuevos robots de Eurobot. Se trata de un sistema embebido basado en \emph{microcontroladores dsPIC} que reparte sus recursos entre la implementaci�n de la plataforma rob�tica y los sistema mec�nicos.

El desarrollo \emph{software} tambi�n est� basado en una \emph{arquitectura} pensada para ser reutilizada. �sta se organiza en capas y/o m�dulos funcionales con diferentes niveles de abstracci�n. El desarrollo software, incluye adem�s, un \emph{simulador de los robots y del campo de juego}. El entorno de simulaci�n permite visualizar los movimientos de los robots sobre un campo de juego virtual y simular robots oponentes.
 

\textbf{Palabras clave:} \mybookpalabrasclave.

%%% Local Variables:
%%% TeX-master: "../book"
%%% End:


